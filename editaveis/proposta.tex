\chapter{Proposta}
\label{proposta}

Este capítulo descreverá as minúcias da implementação deste trabalho, abrangendo, metodologias, cronogramas, técnicas, tecnologias, arquitetura, simulação e os resultados obtidos.

No apêndice \ref{apendiceA} está localizado a simulação desenvolvida na primeira parte do trabalho. Essa simulação teve o intuito de validar e averiguar a viabilidade da proposta do trabalho.

\section{Metodologia}
\label{metodologia}

O ser humano, desde os seus primeiros anos até seus últimos dias de vida busca por mais conhecimento de alguma forma, sendo as dúvidas e questionamentos as principais causas para descobrir respostas para suas perguntas \cite{Paschoarelli_Medola_Bonfim_2018}. A pesquisa auxilia o ser humano nessa trajetória de busca por mais conhecimento, para solucionar problemas do mundo real \cite{gil:2008}.

Segundo Gil (\citeyear{gil:2002}), o objetivo de pesquisa se divide em três categorias: \textbf{Exploratória}, \textbf{Descritiva} e \textbf{Explicativa}. A primeira possibilita uma maior familiaridade com o contexto da problemática, com o princípio de aprimorar ideias para a solução. A segunda permite a descrição das características de algum determinado fenômeno ou um grupo de seres vivos ou relação entre variáveis. A terceira promove a identificação dos fatores que ocasionam ou ajudam para o acontecimento de determinado fenômeno.

Conforme Freitas (\citeyear{de2013metodologia}), para a abordagem do problema existem dois tipos: \textbf{Quantitativo} e \textbf{Qualitativo}. A primeira considera que tudo pode ser quantificado, isto é, converter informações em números para poder ser classificado e analisado posteriormente, e para isso, é necessário o uso de técnicas estatísticas. A segunda não exige técnicas estatísticas, pois a ideia principal dele é analisar indutivamente os dados, sendo que o processo e seu significado são os pontos de foco.

De acordo com Freitas (\citeyear{de2013metodologia}), sobre a natureza da pesquisa, ela pode ser: \textbf{Básica} e \textbf{Aplicada}. A primeira tem o intuito de produzir novos conhecimentos úteis para o crescimento da ciência, sem ter uma previsão de aplicação prática daquilo que foi pesquisado. A segunda também tem o objetivo de gerar novos conhecimentos, mas visado para a aplicação prática para a solução de problemas específicos da vida real.

Um dos procedimentos de pesquisa que existem é denominado Pesquisa-Ação, que consiste em quando há um interesse coletivo para solucionar um determinado problema, no qual os pesquisadores e os participantes estão envolvidos nesse problema de forma cooperativa ou participativa, com o intuito de solucionar ou no mínimo esclarecer o problema da situação, aumentando o nível de conhecimento dos pesquisadores e dos participantes \cite{de2013metodologia}.

Segundo Thiollent (1997), citado por Costa, Politano e Pereira (\citeyear{Costa}), as etapas da Pesquisa-ação são: \textbf{Diagnóstico}, \textbf{Planejamento da ação}, \textbf{Execução da ação} e \textbf{Avaliação da ação}. A primeira, identificar o problema no contexto que o pesquisador tem interesse em resolver. A segunda, arquitetar ações alternativas para a solução do problema, melhor dizendo, desenvolver um planejamento de ação. A terceira, aplicação do planejamento da fase anterior com o auxílio de um roteiro. A quarta, monitorar as ações implementadas para compreender se os resultados são os esperados.

Para este trabalho foi decidido aplicar como objetivo de pesquisa a categoria exploratória, pois como o conhecimento sobre sistemas de recomendação não eram suficientes a solução do problema apresentado seria necessário adquirir mais conhecimento sobre os devidos conceitos. Assim sendo, houve um empenho maior em realizar pesquisas bibliográficas para achar conteúdos da área e trabalhos semelhantes, podendo ainda ser utilizada dessas pesquisas para explorar ou conhecer novos conteúdos que possam vir a serem necessários \cite{Moretti:2018}.

Para o contexto do presente trabalho foi escolhido a abordagem quantitativa, visto que a ideia principal do mesmo é apresentar resultados de natureza estatística como a métrica CVR (\textit{Conversion Rate}) que será aplicada ao final e abordada mais a frente no trabalho na seção \ref{sr_ml}. Com essa métrica será possível medir o grau de melhoria aplicando-se uma abordagem de recomendação comparado com uma já existente em outro sistema.

Como o objetivo deste trabalho é desenvolver um produto específico (Sistema de Recomendação) para o cliente e colaborador Liva, sendo necessário também obter mais conhecimento sobre o contexto do problema em questão para elaboração da solução adequada, foi escolhido para a natureza de pesquisa a categoria aplicada.

Para o procedimento de pesquisa foi utilizada a pesquisa bibliográfica no início do trabalho para conhecer os conceitos, abordagens e técnicas empregadas nos Sistema de Recomendação atualmente.

A organização para elaboração desse projeto realizou as fases de diagnóstico e planejamento do produto como a primeira etapa, sendo identificada como a parte do TCC1 (Trabalho de Conclusão de Curso 1), enquanto que a fase de execução da ação e avaliação serão executadas na etapa correspondente ao TCC2 (Trabalho de Conclusão de Curso 2) deste projeto.

Na fase de diagnóstico foram realizadas reuniões presenciais e remotas com o gerente de produtos da Liva, sendo efetuados o levantamento de informações do problema. Reuniões presenciais com o orientador também aconteceram envolvendo, principalmente, as questões de pesquisa e objetivos, métodos e prazos para a elaboração do projeto e o posterior desenvolvimento do produto.

A fase de maior duração foi a de planejamento, pois foi realizado uma extensa pesquisa e coleta de referências bibliográficas sobre os Sistemas de Recomendação, a fim de se decidir quais abordagens são interessantes para o contexto do trabalho, além da pesquisa sobre soluções parecidas. Com isso foi construído o referencial teórico e efetuado o planejamento da proposta sobre como seria a arquitetura de software (produto), metodologias de pesquisa e de desenvolvimento, levantamento de funcionalidades (requisitos), protótipo de baixa e alta fidelidade e as tecnologias envolvidas.

A execução da ação será a implementação do sistema de recomendação como um serviço a parte dos serviços da \textit{startup} Liva, sendo sintetizadas as definições desse trabalho no quadro \ref{quadro:metodologia_pesquisa}.

\begin{quadro}[h!]
\centering
\caption[Escolha de metodologia de pesquisa]{Escolha de metodologia de pesquisa.}
\label{quadro:metodologia_pesquisa}
\setlength{\tabcolsep}{20pt}
\begin{tabular}{cccc}
\hline
\textbf{\begin{tabular}[c]{@{}c@{}}Quanto à\\ abordagem\end{tabular}} & \textbf{\begin{tabular}[c]{@{}c@{}}Quanto ao\\ objetivo\end{tabular}} & \textbf{\begin{tabular}[c]{@{}c@{}}Quanto à\\ natureza\end{tabular}} & \textbf{\begin{tabular}[c]{@{}c@{}}Quanto ao\\ procedimento\end{tabular}} \\ \hline
\begin{tabular}[c]{@{}c@{}}Pesquisa\\ quantitativa\end{tabular} & \begin{tabular}[c]{@{}c@{}}Pesquisa\\ exploratória\end{tabular} & \begin{tabular}[c]{@{}c@{}}Pesquisa\\ aplicada\end{tabular} & \begin{tabular}[c]{@{}c@{}}Pesquisa bibliográfica\\ e Pesquisa-ação\end{tabular} \\ \hline
\end{tabular}
\end{quadro}

Cada tarefa do fluxo de atividades do TCC1 (Figura \ref{fig:fluxo_atividade_tcc1}) será sucintamente descrito a seguir:

\begin{itemize}
    \item \textbf{Definir tema para o TCC:} é discutido a área em que será feita o trabalho até chegar em uma decisão final;

    \item \textbf{Levantar referencial teórico inicial:} é estudado de forma introdutória sobre o tema escolhido para averiguar se é possível realizar a proposta do trabalho;

    \item \textbf{Elaborar proposta da monografia:} estabelecer escopo do trabalho, objetivo, questões de pesquisa, justificativa e metodologia a ser usada, podendo haver alterações de acordo com o andamento do trabalho;

    \item \textbf{Estabelecer metodologia de pesquisa:} após o início do estudo do tema selecionado anteriormente, é possível definir quais metodologias de pesquisa serão usadas;

    \item \textbf{Pesquisar sobre sistema de recomendação:} envolve a pesquisa de artigos, revistas e de trabalhos científicos sobre sistema de recomendação de forma detalhada para servir de fundamento para os objetivos acertados anteriormente;

    \item \textbf{Estabelecer suporte tecnológico:} seleção das tecnologias que serão usadas para o desenvolvimento do trabalho;

    \item \textbf{Estabelecer metodologia de desenvolvimento:} seleção e planejamento da metodologia de desenvolvimento que será utilizada para implementação da proposta;

    \item \textbf{Implementar simulação:} fase de implementação inicial da solução para validação da proposta e identificação de possíveis riscos que possam interferir no desenvolvimento do trabalho;

    \item \textbf{Revisar TCC1:} revisar todos os tópicos da parte escrita do trabalho para verificar se é necessário algum ajuste ou complementação;

    \item \textbf{Apresentar TCC1:} apresentação do tema para a banca examinadora com o intuito de aprimorar a proposta, caso necessário, e inicio da segunda etapa (TCC2).

\end{itemize}

\begin{figure}[H]
    \centering
    \includegraphics[scale=0.9]{figuras/proposta/fluxo_atividade_tcc1.png}
    \caption[Fluxo de atividades da primeira etapa (TCC1)]{Fluxo de atividades da primeira etapa (TCC1).}
    \label{fig:fluxo_atividade_tcc1}
\end{figure}

Na Figura \ref{fig:fluxo_atividade_tcc2} pode ser visto o fluxo de tarefas para condução do TCC2 e uma breve descrição de cada tarefa, a seguir:

\begin{itemize}
    \item \textbf{Realizar as devidas correções:} após apresentação para a banca examinadora do TCC1 poderá ser realizados os ajustes e complementações solicitadas;

    \item \textbf{Desenvolver o sistema de recomendação:} implementar o sistema de recomendação usando as três abordagens estudadas (\textit{collaborative filtering, content-based filtering e critiquing-based}), sendo duas delas elaboradas com aprendizado de máquina. Será implementada a parte de virtualização de ambiente completo, tanto do ambiente de desenvolvimento quanto de produção, a integração contínua e o \textit{deploy} (entrega) automático no ambiente de homologação;

    \item \textbf{Implantar sistema de recomendação em ambiente de produção:} após o desenvolvimento do sistema de recomendação proposto no trabalho, será feito implantação do mesmo em ambiente de produção da Liva, para assim ser utilizado pelos usuários.

    \item \textbf{Averiguar resultados obtidos:} com a coleta de dados através da implantação do sistema de recomendação em ambiente de produção, será averiguado com a métrica CVR (seção \ref{nivel3}) a eficácia do sistema proposto;
    
    \item \textbf{Revisar TCC2:} verificar se os objetivos do trabalho foram cumpridos como um todo e revisar todos os tópicos da monografia final;

    \item \textbf{Apresentar TCC2:} apresentação do projeto proposto (etapa TCC1) e o seu desenvolvimento com resultados na etapa final (TCC2).
    
\end{itemize}

\begin{figure}[H]
    \centering
    \includegraphics[scale=0.8]{figuras/proposta/fluxo_atividade_tcc2.png}
    \caption[Fluxo de atividades da segunda etapa do projeto (TCC2)]{Fluxo de atividades da segunda etapa do projeto (TCC2).}
    \label{fig:fluxo_atividade_tcc2}
\end{figure}

%\subsection{Metodologia de desenvolvimento}

Para o presente trabalho foi escolhido como metodologia de desenvolvimento o \textit{Scrum}, metodologia ágil e muito utilizada nos últimos anos pela indústria de desenvolvimento de software, acompanhado com a metodologia Kanban.

\subsection{\textit{Scrum}}
\label{section_scrum}

O \textit{Scrum} possibilita um tipo de gerenciamento organizacional ágil e oferece às organizações de software uma nova e dinâmica gestão para não só sobreviver, mas ter sucesso no setor que constantemente tem mudanças e inovações \cite{ijcf94}.

Sua metodologia é baseada na ideia de que muitos processos no desenvolvimento de software são difíceis de serem previstos, sendo assim, a flexibilidade adotada por essa metodologia ágil facilitaria muito mais a implementação das aplicações \cite{ijcf94}.

Como este trabalho é um software que deve ser entregue com uma alta qualidade esperada pelos superiores da Liva, e deve ser colocado que imprevistos são bem prováveis de acontecer, além de mudanças ao decorrer do trabalho, é necessário empregar um \textit{framework} que atenda a essas características, ou seja, uma metodologia em que seja permitido entregar subconjuntos do projeto a cada período de tempo e que possa ser melhorado no andamento do desenvolvimento junto com um \textit{feedback} contínuo.

O princípio do \textit{Scrum} é entregar produtos de alta qualidade, após vários períodos de desenvolvimento chamados de \textit{Sprints} (ciclos) que geralmente tem duração entre duas e quatro semanas, mas para esse trabalho terá a duração de 1 semana. Em cada início de um ciclo será realizado uma reunião chamada de \textit{Sprint Planning} (planejamento do ciclo), na qual é feito o planejamento do que será implementado e entregue ao cliente. Da segunda \textit{Sprint} em diante é realizado uma outra reunião chamada de \textit{Sprint Review} (revisão do ciclo) que seria a revisão do ciclo passado com o propósito de rever o que foi feito na \textit{Sprint} antes de realizar o \textit{Sprint Planning} \cite{ijcf94}.

Como instrumento para organização do que é necessário fazer, será feito o \textit{Product Backlog} (lista de funcionalidade do produto), que é uma lista com todas as funcionalidades que devem conter com base nos requisitos levantados com o cliente. Cada funcionalidade será agrupada pelo seu nível de prioridade. Essa lista é dinâmica e será atualizada de acordo com o andamento do projeto, identificando mais necessidades, como correções e melhorias.

A lista de funcionalidades que é planejada para ser feita no período da \textit{Sprint} é chamada de \textit{Sprint Backlog} (lista de funcionalidades do ciclo), sendo recomendado possuir pelo menos uma melhoria de alta prioridade ao cliente que foi identificada anteriormente \cite{ijcf94}.

Cada funcionalidade do \textit{Product Backlog} será descrito como \textit{User Story} (estória de usuário), na qual é uma representação de cada necessidade do cliente sob seu ponto de vista \cite{knowledge21:2019}.

\subsection{Kanban}
\label{section_kanban}

Kanban (cartão) será uma outra metodologia ágil utilizada no presente projeto. Ela é muito utilizada ao junto com o \textit{Scrum}, sendo um sistema visual para controlar a gestão de atividades em um processo. O Kanban procura observar com facilidade para controlar o fluxo de trabalho, balancear a ordem dos processos e respeitar a produtividade da equipe \cite{ARTIA:2019}.

O Kanban é divido em três colunas: \textit{to do} (fazer), \textit{doing} (fazendo) e \textit{done} (concluído), mas é possível colocar mais colunas caso seja desejado, para assim, ficar mais relacionado ao contexto do processo. Para o presente trabalho haverá seis colunas: \textit{Product Backlog} (lista de funcionalidades do produto), \textit{Sprint Backlog} (lista de funcionalidades do ciclo), \textit{Doing} (fazendo), \textit{Testing} (teste), \textit{Review} (revisão) e \textit{Done} (concluído). Para cada coluna há cartões (\textit{cards}) que representam uma atividade que está em uma dessas respectivas fases do processo. De acordo com o andamento, os \textit{cards (cartão)} serão movidos para as fases seguintes \cite{ARTIA:2019}.

Para realização dessa metodologia será utilizada a ferramenta ZenHub que será descrita na seção \ref{section_zenhub}.

\subsection{\textit{Product Backlog} inicial}
\label{section_product_backlog}

Para obter um conhecimento inicial sobre o projeto, foram levantado as seguintes \textit{features} (características), \textit{user stories} (estórias de usuário) e \textit{technical stories} (estória técnica):

\begin{itemize}
    \item Características (\textit{Features}):
    \begin{itemize}
        \item \textbf{FE01 - Recomendador baseado em crítica:} possibilitará ao usuário receber recomendações de imóveis baseadas em críticas sobre características de imóveis, feitas pelo próprio usuário;

        \item \textbf{FE02 - Recomendador baseado em aprendizado de máquina:} permitirá ao usuário receber recomendações de imóveis baseado nas ações dele no sítio virtual.
    \end{itemize}
    
    \item Estória do usuário (\textit{User stories}):
    \begin{itemize}
        \item \textbf{FE01US01 - Visualizar perfil de busca:} eu, como usuário, desejo visualizar meu perfil de busca para ter conhecimento de como está configurado;
        
        \item \textbf{FE01US02 - Alterar perfil de busca:} eu, como usuário, desejo alterar meu perfil de busca para ajustar as preferência e receber recomendações mais precisas;
        
        \item \textbf{FE01US03 - Visualizar recomendações na página principal:} eu, como usuário, desejo visualizar os imóveis recomendados com base nas minhas preferências configuradas no perfil de busca;
        
        \item \textbf{FE01US04 - Criticar características da propriedade:} eu, como usuário, desejo criticar as características de um ou vários imóveis para ter recomendações mais adequadas com o que eu desejo;
        
        \item \textbf{FE02US05 - Visualizar recomendações na página do imóvel:} eu, como usuário, desejo visualizar recomendações de imóveis baseadas nas minhas ações no sítio virtal e na página de detalhes do imóvel para diminuir o tempo de procura do imóvel ideal para mim;
    \end{itemize}
    
    \item Estórias técnicas (\textit{Technical stories}):
    \begin{itemize}
        \item \textbf{FE02TS01 - Registrar ações do usuário no sítio virtual:} eu, como desenvolvedor, desejo registrar as ações do usuário no site para poder predizer quais são os imóveis mais prováveis da preferência do usuário.
    \end{itemize}
\end{itemize}

\subsection{Requisitos não funcionais}

Foram selecionados os requisitos não funcionais do sistema de recomendação proposto em categorias. Foram escolhidas as consideradas mais importantes, que são elas: usabilidade, desempenho, segurança, disponibilidade, portabilidade e confiabilidade.

\subsubsection{Usabilidade}

No Sistema de recomendação haverá dois componentes na plataforma \textit{web} Liva. Ambos possuindo um design atrativo para intrigar o usuário ao uso, e também responsivo para poder ser visualizado em qualquer tela, independente do dispositivo ser um computador, \textit{smartphone} ou \textit{tablet}. Deverá ser também intuitivo, contendo ícones para cada atributo do imóvel, instrução de como utilizar, mensagem informativa de sucesso, falha ou redirecionamento de página e botões de aumento e diminuição da quantidade numérica nos campos do formulário.

\subsubsection{Desempenho}
\label{desempenho}

No modulo de Aprendizado de Máquina, devido ao alto volume de dados e processamentos demandados do treino e predição do modelo, na qual é necessário ser processado em menos de 10 segundos para o usuário, são demandadas instancias de servidores adequadas ao contexto. Para isso a instancia a ser escolhida deverá ter os seguintes recursos computacionais:

\begin{itemize}
    \item 2 Unidades de vCPU
    \item 8 gigas de memória ram
    \item 20 gigas de espaço em disco (crescendo horizontalmente se necessário)
\end{itemize}

\subsubsection{Segurança}

A API do sistema poderá ser acessada por sistemas externos de domínio da Liva. Este acesso deve ser seguro, com autenticação em nível do servidor e em nível da aplicação. Os usuários devem ter conta na plataforma da Liva, para assim, terem acesso as funcionalidades do sistema de recomendação.

\subsubsection{Disponibilidade}

O Sistema de recomendação precisa funcionar 24 x 7 (vinte e quatro horas por dia, sete dias por semana) na operação da Liva, pois há constantemente clientes acessando o sistema. Dessa forma é necessário que a infraestrutura suporte esse período e cressa horizontalmente em seus recursos de acordo com o que é demandado, para assim se manter sempre de pé.

\subsubsection{Portabilidade}

O sistema proposto deverá rodar nos navegadores mais populares, como: Mozilla Firefox, Google Chrome, Microsoft Edge e Opera. Navegadores mais antigos como o Internet Explorer podem sofrer de estabilidade na utilização do site, por não haver suporte com ES5 que é requisito da tecnologia da plataforma.

\subsubsection{Confiabilidade}

O Sistema de recomendação deve ser um sistema eficiente que apresente sempre as recomendações mais adequada possíveis, sendo representantes de seu perfil. Dessa forma o modelo de Aprendizado de Máquina deve apresentar boas métricas de qualidade, e assim apresentar boas recomendações, fazendo os clientes se sentirem com mais confiança em relação ao sistema.

\subsection{Papéis}

\textbf{\textit{Product Owner}} (Dono do Produto). É responsável por potencializar o retorno sobre o ROI (\textit{Return over Investment} - Retorno Sobre Investimento), e para isso é necessário que esse papel identifique as características do produto, decidindo quais são as funcionalidades com mais prioridades, criando uma lista ordenada com base na prioridade para a próxima \textit{Sprint} e re-priorizando e refinando continuamente essa lista. De forma resumida, o dono do produto é responsável pelos lucros e perdas do produto \cite{Sutherland}.

Esse papel será desempenhado pelo gerente de produtos da Liva, pois ele tem a visão de produto deste trabalho que está sendo desenvolvido, além de também ter conhecimento de Engenharia de Software.

\textbf{\textit{Scrum Master}} (Mestre Scrum). Tem como atividades ensinar o \textit{Scrum} para a equipe do produto e aplicá-lo de fato. Seu objetivo principal é ajudar a equipe a alcançar o sucesso, protegendo-os de possíveis interferências externas. Além disso, ele também educa e  orienta o dono do produto e a todos do grupo no uso da metodologia ágil. Quando necessário, ajuda a liderar a organização nas difíceis mudanças fundamentais para o sucesso do desenvolvimento ágil \cite{Sutherland}.

Quem irá desempenhar esse papel será o professor e orientador, por ser quem vai acompanhar e auxiliar o andamento do trabalho.

\textbf{Time desenvolvimento}. Serão as pessoas que farão a implementação do produto, fazendo entregas a cada \textit{Sprint} \cite{Sutherland}. Esse papel será desempenhado pelos orientandos do trabalho (estudantes envolvidos com o TCC), por possuírem conhecimento técnico sobre as tecnologias, metodologias, além do contexto do trabalho.

\subsection{Fluxo de desenvolvimento}
\label{fluxo_desenvol}

Na atividade "Desenvolver o sistema de recomendação" da Figura \ref{fig:fluxo_atividade_tcc2}, será realizado o fluxo de desenvolvimento representado a seguir (Figura \ref{fig:fluxo_scrum}).

\begin{figure}[H]
    \centering
    \includegraphics[scale=0.6]{figuras/proposta/fluxo_scrum.png}
    \caption[Fluxo de desenvolvimento do trabalho]{Fluxo de desenvolvimento do trabalho.}
    \label{fig:fluxo_scrum}
\end{figure}

A descrição resumida de cada tarefa se encontra a seguir:

\begin{itemize}
    \item \textbf{Identificar \textit{features} (características):} averiguar quais são os módulos de requisitos do projeto existentes.
    
    \item \textbf{Identificar \textit{user stories} (estórias de usuário) e \textit{technical stories} (estórias técnicas):} identificar as estórias de cada \textit{feature} (característica) definida anteriormente e prioriza-las com o nível de importância para o cliente;
    
    \item \textbf{\textit{Sprint planning} (planejamento do ciclo) :} planejar quais estórias serão desenvolvidas na \textit{Sprint}, sendo seguida a ordem de prioridade descrita no \textit{product backlog} (lista de funcionalidades do produto);
    
    \item \textbf{Desenvolver \textit{stories} (estórias):} desenvolver as estórias planejadas para aquela \textit{Sprint} até a conclusão das mesmas. A conclusão se define em implementar, testar e verificar se está de acordo com os critérios de aceitação da estória definida;
    
    \item \textbf{\textit{Sprint review} (revisão de ciclo):} verificar quais estórias foram realmente feitas e quais estão pendentes, em que se houve a necessidade de criar novas, seja estória de usuário ou estória técnica.
\end{itemize}

\section{Cronograma}

Foi planejado dois cronogramas para serem seguidos no TCC1 e no TCC2 com intuito de dividir as atividades durante o semestre. O primeiro cronograma é referente a primeira parte do projeto que abrange todas as atividades referentes ao TCC1 e o segundo cronograma é referente as atividades que foram propostas para o TCC2. A utilização do cronograma se faz necessária para ter um gerenciamento de tempo das atividades que precisam ser realizadas para a conclusão deste trabalho. Pode ser ver nas tabelas a seguir (Tabela \ref{tab:cronograma_tcc1} e Tabela \ref{tab:cronograma_tcc2}) como será organizado as atividades do TCC1 e do TCC2 respectivamente.

\begin{table}[H]
\centering
\caption[Cronograma do TCC1]{Cronograma do TCC1.}
\label{tab:cronograma_tcc1}
\begin{tabular}{lccccc}
\hline
\textbf{} & \textbf{Ago} & \textbf{Set} & \textbf{Ou} & \textbf{Nov} & \textbf{Dez} \\ \hline
Escrever monografia & X & X & X & X &  \\ \hline
Estudo inicial do tema & X &  &  &  &  \\ \hline
Definir de escopo e objetivo & X &  &  &  &  \\ \hline
Desenvolver proposta &  & X & X &  &  \\ \hline
Determinar metodologias &  &  & X &  &  \\ \hline
Revisar TCC1 &  &  & X & X &  \\ \hline
Apresentar TCC1 &  &  &  &  & X \\ \hline
\end{tabular}
\end{table}

\begin{table}[H]
\centering
\caption[Cronograma do TCC2]{Cronograma do TCC2.}
\label{tab:cronograma_tcc2}
\begin{tabular}{lccccc}
\hline
\textbf{} & \textbf{Mar} & \textbf{Abr} & \textbf{Mai} & \textbf{Jun} & \textbf{Jul} \\ \hline
Efetuar revisão da banca & X &  &  &  &  \\ \hline
Desenvolver o sistema de recomendação & X & X & X &  &  \\ \hline
Implantar sistema de recomendação \\ em ambiente de produção &  &  & X &  &  \\ \hline
Averiguar resultados obtidos &  & X & X &  & \\ \hline
Revisar TCC2 &  & X & X &  & \\ \hline
Apresentar TCC2 &  &  & X &  & \\ \hline
\end{tabular}
\end{table}

\section{Sistema de recomendação proposto}
\label{section_sr}

O sistema proposto consiste de um novo sistema de recomendação para a plataforma Web Liva.vc. Esse sistema de recomendação seguirá duas abordagens: sistema de recomendação baseado em crítica (seção \ref{Critiquing-based} desse trabalho) e sistema de recomendação baseado em aprendizado de máquina (seção \ref{machineLearning} desse trabalho). Por se tratar do uso de duas abordagens, isso faz com que se torne um sistema de recomendação híbrido (seção \ref{Hybrid} desse trabalho). Para fazer a combinação dessas duas abordagens é utilizada a técnica de hibridação \textit{mixed} (mista), que como já abordado, refere-se a recomendações de diferentes recomendadores que são apresentadas juntas.

Assim, o sistema proposto irá combinar essas duas abordagens que irão englobar diferentes páginas do site da Liva. Existem duas principais páginas que irão ser atualizadas para o novo sistema de recomendação: página principal do cliente (Figura \ref{fig:pagina_principal}) e página de detalhes de uma propriedade (Figura \ref{fig:pagina_detalhes})

\begin{figure}[H]
    \centering
    \includegraphics[scale=0.33]{figuras/proposta/pagina_principal.png}
    \caption[Página principal do cliente]{Página principal do cliente. Fonte:  \cite{Liva:2019}.}
    \label{fig:pagina_principal}
\end{figure}

\begin{figure}[H]
    \centering
    \includegraphics[scale=0.23]{figuras/proposta/pagina_detalhes.png}
    \caption[Página de detalhes da propriedade]{Página de detalhes da propriedade. Fonte: \cite{Liva:2019}.}
    \label{fig:pagina_detalhes}
\end{figure}

Na página principal existe uma ferramenta de busca denominada “Perfil de Busca” e pode ser acessada por uma modal, como mostrado na Figura \ref{fig:filtro_busca}.

\begin{figure}[H]
    \centering
    \includegraphics[scale=0.6]{figuras/proposta/filtro_busca.png}
    \caption[Filtro de busca]{Filtro de busca. Fonte: \cite{Liva:2019}.}
    \label{fig:filtro_busca}
\end{figure}

Aproveitando-se dessa ferramenta de busca para implementação do sistema de recomendação baseado em crítica será necessário alguns ajustes. Primeiramente, serão colocados novos campos para determinar a importância de cada característica de preferência do usuário. Além disso, será adicionado um novo campo de quantidade referente a número de banheiros. Os campos que representam a quantidade das características serão alterados para quantidades fixas e não mínimas. A localidade irá se referir somente a um bairro. Essas alterações podem ser vista no protótipo elaborado na Figura \ref{fig:prototipo_search_profile}.

\begin{figure}[H]
    \centering
    \includegraphics[scale=0.6]{figuras/proposta/prototipo_search_profile.jpg}
    \caption[Protótipo do filtro de busca]{Protótipo do filtro de busca.}
    \label{fig:prototipo_search_profile}
\end{figure}

Ainda na página principal é possível observar a existem de um \textit{feed} (lista com conteúdo atualizado periodicamente), em que se encontram as recomendações do atual recomendador usado pela Liva. Esse \textit{feed} servirá para visualizar as recomendações do novo recomendador baseado em crítica.

A página de detalhes de uma propriedade é acessível por meio de um clique em sua imagem no \textit{feed} da página principal. Nela é possível visualizar todas as características da propriedade. Além disso, é possível entrar em contato com o corretor de duas formas diferentes: pedindo mais informações da propriedade e agendando uma visita. Para o recomendador baseado em crítica será atualizado o componente de visualização de características da propriedade para que possam ser feitas críticas pelo usuário sobre essas características disponibilizadas para o acesso. Esse novo componente de crítica pode ser observado pelos protótipos mostrados nas Figuras \ref{fig:prototipo_critico1}, \ref{fig:prototipo_critico2}, \ref{fig:prototipo_critico3} e \ref{fig:prototipo_critico4}, respectivamente.

\begin{figure}[H]
    \centering
    \includegraphics[scale=0.5]{figuras/proposta/prototipo1.jpg}
    \caption[Detalhes do imóvel com menu de crítica parte 1]{Detalhes do imóvel com menu de crítica parte 1.}
    \label{fig:prototipo_critico1}
\end{figure}

\begin{figure}[H]
    \centering
    \includegraphics[scale=0.5]{figuras/proposta/prototipo2.jpg}
    \caption[Detalhes do imóvel com menu de crítica parte 2]{Detalhes do imóvel com menu de crítica parte 2.}
    \label{fig:prototipo_critico2}
\end{figure}

\begin{figure}[H]
    \centering
    \includegraphics[scale=0.5]{figuras/proposta/prototipo3.jpg}
    \caption[Detalhes do imóvel com menu de crítica parte 3]{Detalhes do imóvel com menu de crítica parte 3.}
    \label{fig:prototipo_critico3}
\end{figure}

\begin{figure}[H]
    \centering
    \includegraphics[scale=0.5]{figuras/proposta/prototipo4.jpg}
    \caption[Detalhes do imóvel com menu de crítica parte 4]{Detalhes do imóvel com menu de crítica parte 4.}
    \label{fig:prototipo_critico4}
\end{figure}

\begin{figure}[H]
    \centering
    \includegraphics[scale=0.5]{figuras/proposta/prototipo5.jpg}
    \caption[Detalhes do imóvel com menu de crítica parte 5]{Detalhes do imóvel com menu de crítica parte 5.}
    \label{fig:prototipo_critico5}
\end{figure}

\begin{figure}[H]
    \centering
    \includegraphics[scale=0.5]{figuras/proposta/prototipo6.jpg}
    \caption[Detalhes do imóvel com menu de crítica parte 6]{Detalhes do imóvel com menu de crítica parte 6.}
    \label{fig:prototipo_critico6}
\end{figure}

O usuário poderá criticar uma propriedade escolhendo entre manter e mudar os valores para cada característica. Após as críticas serem atribuídas as características de uma propriedade, o perfil de busca desse usuário será aprimorado, gerando novas recomendações no \textit{feed} da página principal.

Ainda referente à página de detalhes, será adicionado também um novo componente na lateral direita para a visualização das recomendações do módulo de aprendizagem de máquina. Esse componente será a representação de cinco propriedades contendo todas suas características principais. Esse componente está representado na Figura \ref{fig:prototipo_recommender_system}.

\begin{figure}[H]
    \centering
    \includegraphics[scale=0.53]{figuras/proposta/prototipo_recommender_system.png}
    \caption[Protótipo do carrossel de recomendações]{Protótipo do carrossel de recomendações.}
    \label{fig:prototipo_recommender_system}
\end{figure}

O recomendador do módulo de aprendizado de máquina terá como dados de entrada registros de ações de usuários na plataforma, como: cliques ou descartes em propriedades no \textit{feed}, alterações no perfil de busca e até mesmo em requisições de contato com o cliente. Dessa forma, as recomendações serão realizadas em tempo real, no momento em que um usuário entra na página de detalhes de uma propriedade.

\subsection{Arquitetura}

Para este trabalho de conclusão de curso será desenvolvido uma aplicação web como um serviço de sistema de recomendação para o sítio virtual da Liva.vc. Sua principal funcionalidade será fornecer recomendações de propriedades para usuários que sejam clientes de imobiliárias vinculadas a Liva.

Para apresentar a arquitetura do projeto, houve uma divisão em três níveis, em que serão apresentadas as características e os motivos de suas implementações. Começando com o primeiro nível uma visão mais generalizada do sistema é apresentada. Em seguida serão apresentadas características mais detalhadas nos seguintes níveis, a fim de melhor esclarecer o projeto proposto que se deseja implementar. 

\subsubsection{Nível 1}

Considerando o sistema de recomendação como uma caixa preta, a Figura \ref{fig:sr_nivel1} demonstra a interação da Liva com o novo sistema de recomendação, descrevendo suas entradas e saídas de dados.

\begin{figure}[H]
    \centering
    \includegraphics[scale=0.4]{figuras/proposta/sr_nivel1.png}
    \caption[Sistema de recomendação como caixa preta]{Sistema de recomendação como caixa preta.}
    \label{fig:sr_nivel1}
\end{figure}

Para os usuários terem acesso ao ambiente virtual do Liva eles devem, primeiramente, ter interagido com uma outra plataforma de vendas de imóveis tais como Zap Imóveis ou Olx. Essa interação refere-se a um \textit{lead}, que é uma requisição de informações  a um corretor a respeito de um imóvel de uma imobiliária vinculada a Liva. Quando usuários acessam a Liva pela primeira vez eles já possuem diversos dados advindos de seu primeiro \textit{lead}. Com isso, logo no primeiro acesso à página principal o usuário já tem os dados de entrada necessários para receber recomendações. A partir desse momento dados de navegação relevantes começam a ser registrados, tais como: trocar seu Perfil de Busca, clicar para visualizar propriedades, descartar propriedades, requisitar contato com o corretor (agendar visita ou pedir informações da propriedade) e criticar características das propriedades. Todo novo registro é enviado ao sistema de recomendação, que por sua vez, apresentará novas recomendações. Dessa forma, a cada novo registro as recomendações são reprocessadas e tornam-se eventualmente mais enriquecidas para fornecer outras recomendações ao usuário.

O motivo de se criar um sistema de recomendação baseado nos dados do processo de consulta do usuário para encontrar uma propriedade adequada é justificado por eles serem valiosos representantes de sua preferência. Normalmente, um usuário consulta diversas vezes e lê várias páginas antes de encontrar a propriedade certa. Esse processo é considerado muito trabalhoso e pode gerar frustrações aos usuários a ponto de fazê-los desistirem. O foco é melhorar a experiência do usuário no sistema durante esse processo, gerando recomendações ainda mais precisas a cada nova interação.

\subsubsection{Nível 2}
\label{nivel2}
Diminuindo ainda mais o nível de abstração da arquitetura, a Figura \ref{fig:sr_nivel2} apresenta os serviços, suas tecnologias e interações do projeto.

\begin{figure}[H]
    \centering
    \includegraphics[scale=0.4]{figuras/proposta/sr_nivel2.png}
    \caption[Arquitetura geral do sistema divida em serviços]{Arquitetura geral do sistema divida em serviços.}
    \label{fig:sr_nivel2}
\end{figure}

A arquitetura existente no sistema da Liva segue o estilo Cliente-Servidor. Seu cliente é desenvolvido na linguagem JavaScript e com o \textit{framework} React. Em seu \textit{back-end}, Liva API (Application Programming Interface - Interface de Programação de Aplicação) na linguagem Ruby com suporte do \textit{framework} Ruby on Rails, é integrado a um sistema gerenciador de banco de dados PostgreSQL.

A fim de desenvolver um sistema como um serviço que possa ser utilizado pela Liva, sem interferir diretamente em sua estrutura, será implantado uma arquitetura em que o sistema de recomendação a ser implementado é separado do restante do sistema, como demonstrado na Figura \ref{fig:sr_nivel2}, destacado em azul. O sistema é composto por um cliente que é representado pelos novos componentes de interface atribuídos ao \textit{front-end} da Liva. Esses componentes são: nova ferramenta de busca com o componente de importância para cada atributo (Figura \ref{fig:filtro_busca}), componente de crítica (Figura \ref{fig:prototipo_critico1}) e por fim o componente de recomendação, responsável por listar recomendações advindas do módulo de aprendizado de máquina (Figura \ref{fig:prototipo_recommender_system}). Eles se comunicarão através do protocolo HTTP (Protocolo de Transferência de Hipertexto) com a API (Interface de Programação de Aplicações) de recomendação e com a tecnologia Amazon Kinesis, que transmitirá dados para o Redis em tempo real.

A API de recomendação será desenvolvida em Python com o \textit{framework} Django REST e se comunicará com outras três tecnologias: Redis, PostgreSQL e Amazon S3. A escolha dessa tecnologia foi feita pelo fato de que existem diversas bibliotecas e \textit{frameworks}c desenvolvidos em Python que dão suporte para suprir todos os requisitos do sistema a ser desenvolvido, como por exemplo, o XGboost, Scikit-Learn, Numpy, Pandas e o Pickle para solução do módulo de aprendizado de máquina, além do Django REST que possibilita o desenvolvimento de uma API REST.

Como haverá novos componentes na interface gráfica referentes ao sistema de recomendação, eles serão desenvolvidos na linguagem JavaScript com o suporte do \textit{framework} React já utilizado no ambiente da Liva, consequentemente, sendo facilmente integrável.

O sistema proposto é composto por um banco de dados de grande escala para o armazenamento de registros dos usuários e propriedades em memória RAM (Redis) para otimização no tempo de respostas de requisições a respeito da recomendação. Além disso há a existência de um banco relacional para \textit{backup} (cópia de segurança) dos dados e para salvar novos dados de treinos do modelo de aprendizado de máquina (PostgreSQL). O Redis foi escolhido para esse propósito, por ser rápido e por ser um banco de dados de arquitetura valor-chave, o que facilita no mapeamento dos dados. Essas escolhas são essenciais para o sistema de recomendação pois as recomendações das propriedades precisam ser geradas rapidamente para o usuário. Segundo Li et al. (\citeyear{Summo:2017}) na plataforma Suumo.jp eles calculam de dez a vinte segundos para um usuário ler as informações da propriedade para somente em seguida visualizarem as recomendações na página de detalhes. 

O Amazon S3 servirá para guardar os dados de treino e o modelo treinado do módulo de aprendizagem de máquina. Essa tecnologia é utilizada para melhorar no tempo de treino do modelo, que será feito em lotes, e a fácil distribuição do modelo.

O sistema também é composto por um serviço de \textit{upload} (modificação) de dados (\textit{Data Updater}), que será desenvolvido na linguagem Javascript com o ambiente de execução NodeJS, capaz de ler o banco de dados da Liva e atualizar o banco do sistema de recomendação. Esses dados são basicamente referentes a propriedades.

O sistema de recomendação e todos os serviços que o envolvem serão implantados em ambiente de produção por meio de contêineres, possibilitado pela tecnologia Docker e serão orquestrados pelo serviço do AWS, Amazon ECS, já utilizado na Liva.
	
\subsubsection{Nível 3}
\label{nivel3}

O sistema de recomendação a ser desenvolvido foi dividido em dois módulos. Cada módulo refere-se a um tipo diferente de solução para recomendar itens. Como apresentado anteriormente (seção \ref{section_sr}), o sistema recomendará propriedades para os usuários na página principal do sítio virtual da Liva, no denominado \textit{feed}, baseado na ferramenta de busca, o Perfil de Busca, e as críticas feitas sobre as propriedades pelos usuários. Essa solução refere-se a técnica de crítica de exemplo (\textit{example-critiquing}) como apresentado na seção \ref{Critiquing-based}. A outra solução baseia-se na área de aprendizado de máquina, apresentada na seção \ref{machineLearning}, seguindo o modelo utilizado no site Suumo.jp \cite{Summo:2017}, mas adaptado ao contexto da Liva. Esse módulo diz a respeito das recomendações apresentadas na página de detalhes das propriedades.

\subsection{Sistema de recomendação baseado em Aprendizado de Máquina}
\label{sr_ml}

A escolha da abordagem de aprendizado de máquina justifica-se pela experiência descrita a seguir, feita na plataforma japonesa Suumo.jp \cite{Summo:2017}, que assim como a Liva, categoriza-se como um ambiente virtual de \textit{e-commerce} imobiliário.
a
% O motivo de literaturas referentes aos sítios virtuais imobiliários que utilizam esse tipo de abordagem serem escassos acontece por existirem poucas avaliações sobre produtos como propriedades, por pouca recorrência de compra. Li et al. (\citeyear{Summo:2017}) cria um modelo utilizando-se de dados de entrada advindos de registros de consulta dos usuários em tempo real, como por exemplo, clicar/visualizar propriedades e pesquisar por propriedades em diferentes condições, gerando assim, incrementalmente, uma base consistente de dados representativas da preferência dos usuários.

Li et al. (\citeyear{Summo:2017}) demonstra que para seu contexto os registros de consulta dos usuários como entrada apresenta melhor performance no modelo de aprendizado de máquina quando comparado com outras abordagens. Para fazer essa comparação ele utiliza uma métrica chamada \textit{Conversion Rate} (CVR - Taxa de Conversão). CV ou \textit{Conversion} (Conversão) refere-se a requisição de informação para um corretor de uma propriedade feita por um usuário. O CVR se dá pela quantidade requisições feitas em propriedades recomendadas (A) dividido por todas as recomendações clicadas (B).

\begin{equation}
    CVR=\frac{A}{B}
\end{equation}

No sítio virtual Suumo.jp, pioneira no uso de sistema de recomendações aplicado ao contexto de imóveis, foi primeiramente desenvolvido um modelo híbrido de filtragem colaborativa com filtragem baseada em conteúdo, descritos na seção \ref{Collaborativefiltering} e \ref{Contentbasedfiltering} respectivamente. Os resultados apresentaram uma melhora de 25\% na taxa de cliques em propriedades em geral. Em seguida, foi implementado um modelo incremental de filtragem colaborativa restrito a algumas páginas para fazer uma comparação com o primeiro modelo. O segundo modelo produziu uma melhoria de 20\% no CVR. O problema desse tipo de abordagem é que, como já apresentado anteriormente (seção \ref{Collaborativefiltering}), para que um item seja recomendado é preciso que muitas interações de usuários sejam feitas, ou seja, muitas visualizações, o que faz com que novas propriedades tenham menos chance de serem recomendadas. O recomendador baseado em aprendizado de máquina proposto ao final apresentou uma melhora de 250\% na performance. Esse resultado demonstra que mesmo as ações mais sutis podem refletir nas preferências de um usuário e têm um efeito positivo na previsão de suas necessidades.

\subsubsection{\textit{Design} do Modelo}
\label{design_modelo}

Com a finalidade de demonstrar a arquitetura do módulo de aprendizado de máquina e suas principais características, a Figura \ref{fig:sr_nivel3} apresenta o processo de recomendação mais detalhado dentro da API e as interações com as outras tecnologias.

\begin{sidewaysfigure}
    \centering
    \includegraphics[scale=0.4]{figuras/proposta/sr_nivel3.png}
    \caption[Arquitetura do módulo de Aprendizado de Máquina]{Arquitetura do módulo de Aprendizado de Máquina.}
    \label{fig:sr_nivel3}
\end{sidewaysfigure}

Baseado na arquitetura da Figura \ref{fig:sr_nivel3}, primeiramente é passado via requisição o identificador único (chave primária) do usuário para o \textit{endpoint} (endereço de rede para acesso do cliente) de recomendação, no momento em que um usuário entra na página de detalhes de uma propriedade, a fim de ser gerada as recomendações. Com o ID do usuário é possível recuperar os seguintes três tipos de variáveis:

\begin{itemize}
    \item \textbf{Última busca realizada pelo usuário (BU):} refere-se ao atual Perfil de Busca do usuário. Quando um usuário clica para visualizar uma propriedade, ela está sobre diversas condições de busca, como preço mínimo e máximo, tipo de imóvel, quantidade de quartos, dentre outras condições. Essas informações são guardadas na tabela LAST\_SEARCH\_PROFILE\_HISTORY (Último Histórico do Perfil de Pesquisa);

    \item \textbf{Visualizações recentes em propriedades feitas pelo usuário (VU):} quando um usuário clica para visualizar os detalhes de uma propriedade. A cada clique é salvo um novo registro na tabela VIEW\_PROPERTY\_HISTORY (Visão do Histórico de Propriedade). São salvos no máximo 100 registros dos últimos 3 meses para cada usuário. Esses registros são compostos de informações como preço, área, quantidade de quartos, dentre outras características de um imóvel;

    \item \textbf{Propriedade candidata a recomendação (PC):} refere-se às características da propriedade candidata a recomendação como preço, área, quantidade de quartos, dentre outras características. As propriedades são mantidas na tabela PROPERTIES (propriedades).
\end{itemize}

    Para a entrada do modelo de aprendizado de máquina existe um conjunto de {BU, VU, PC} e a saída é a pontuação (\textit{score}) que reflete a probabilidade de CV, ou seja, a probabilidade de um usuário requisitar contato a um corretor de uma propriedade em específico.

\begin{equation}
    \{BU,VU,PC\}  \xrightarrow{F} \mathbb{R}
\end{equation}

Em que F é o modelo que atribui um \textit{score} para um registro composto por BU, VU e PC. Um maior score significa maior chance de um CV. O conjunto de dados de treino é composto por BU, VU e as características da propriedade de cada CV realizado ou descarte de uma propriedade do \textit{feed} referente a todos os usuários, ou seja, a variável alvo nesse caso é o CV, podendo ser 0 ou 1. O algoritmo de aprendizagem de máquina vai construir um modelo capaz de distinguir entre CV = 1 e CV = 0 . No caso de um usuário descartar uma propriedade é considerado CV = 0.

A cada nova interação de descarte ou requisição de informação de uma propriedade os dados BU, VU e as características da propriedade em foco são salvos na tabela NEW\_TRAINING\_DATA (Novos Dados de Treinamento). O treino é realizado por lotes agendados diariamente. Toda vez que o \textit{script} de treino é executado, primeiramente, é verificado a existência de um arquivo .csv na Amazon S3 e em seguida é feita uma integração com os dados de treino da tabela. Logo em seguida é realizada a otimização de hiper-parâmetros com a técnica \textit{Grid Search} (pesquisa em grade). O modelo é treinado e salvo como arquivo .sav na Amazon S3.

Toda vez que um usuário alcança a página de detalhes de uma propriedade é recuperado seu BU e VU (Tabela \ref{tab:conjunto_dados}). Dessa forma, é apurado o \textit{score} para cada candidato PC. Os candidatos são escolhidos baseados na mesma localidade da presente propriedade visualizada pelo usuário. Após realizado o \textit{score} é feita uma ordenação dessas propriedades e em seguida são apresentadas as cinco primeiras ao usuário.

\begin{table}[H]
\centering
\caption[Exemplo de conjunto de dados de treino]{Exemplo de conjunto de dados de treino.}
\label{tab:conjunto_dados}
\begin{tabular}{ccccccc}
\hline
\textbf{\begin{tabular}[c]{@{}c@{}}Preço\\ máximo (BU)\end{tabular}} & \textbf{...} & \textbf{\begin{tabular}[c]{@{}c@{}}Quantidade\\ de\\ quartos 1\\ (VU)\end{tabular}} & \textbf{Preço 2 (VU)} & \textbf{\begin{tabular}[c]{@{}c@{}}Quantidade de\\ banheiro 2 (VU)\end{tabular}} & \textbf{...} & \textbf{CV} \\ \hline
400000 & ... & 1 & 150000 & 2 & ... & 1 \\ \hline
100000 & ... & 3 & null & null & ... & 0
\end{tabular}
\end{table}

Para o presente problema de classificação foi escolhido o modelo de GBDT (\textit{Gradient Boosting Decision Tree} - Árvore de Decisão com Aumento de Gradiente), discutido na seção \ref{ensemble}. Como a característica do conjunto de dados torna possível a existência de colunas com dados ausentes, é muito importante a utilização de um método capaz de lidar com tal problema bem, além de manter uma alta precisão. Por ser um modelo baseado em árvore de decisão, ele consegue lidar bem com atributos em seu valor bruto, o que é preferível em questões de manutenibilidade e performance, pois não necessitará de pré-processamento. Além disso, a tecnologia usada para o modelo lida naturalmente com dados faltantes. Segundo Synced (\citeyear{Synced:2017}), internamente, o XGBoost aprenderá automaticamente qual é a melhor direção a seguir quando um valor estiver faltando. Equivalentemente, isso pode ser visto como "aprender" automaticamente qual é o melhor valor de imputação para valores faltantes com base na minimização da perda do treinamento.

Os três tipos de variáveis apresentadas anteriormente para entrada do modelo são relacionadas a dois tipos matemáticos, numéricos e categóricos. Para os dados numéricos como preço ou quantidade de quartos, por exemplo, foi decidido manter seus valores em sua forma bruta, por questões de manutenibilidade e performance. Para as variáveis categóricas, algumas colunas são excluídas, como o identificador único do usuário, que não é útil para a predição. A localidade será trocada pela sua latitude e longitude e o tipo de imóvel será trocado por variáveis fictícias binárias (0 e 1).

No Suumo.jp, Li et al. (\citeyear{Summo:2017}) encontra uma distribuição de tuplas no conjunto de dados de treino entre CV = 1 e CV = 0, para cada uma linha com CV = 1 são gerada dez linhas com CV = 0. Em seu modelo é proposto dez linhas CV = 0 para dez propriedades aleatórias a cada um CV = 1. Dado que no modelo proposto para esse projeto se tem a interação de descarte de uma propriedade, foi decidido não utilizar essa proporção e aleatoriedade pois isso fere com a realidade dos dados dentro do contexto.

Li et al. (\citeyear{Summo:2017}) afirma que as variáveis categóricas devem ser tratadas em um modelo diferente das variáveis numéricas. Isso porque, como descrito anteriormente na seção \ref{machineLearning}, na construção da árvore de decisão, o algoritmo de ramificação se concentra na divisão de variáveis numéricas, a frequência de aparecimento de variáveis categóricas igualmente importantes para os usuários é relativamente reduzida, ou seja, as árvores de decisão vão favorecer as ramificações com variáveis numéricas.

Dito isso Li et al. (\citeyear{Summo:2017}) propõe a separação de dois modelos, um somente com variáveis categóricas e o outro com numéricas. Para fazer a junção dos \textit{scores} (pontuações) gerados ele propõe a utilização de uma função de soma linear.

Como há uma quantidade muito pequena de características categóricas para o modelo proposto, não será feito a partição entre categóricas e numéricas.

\subsection{Sistema de recomendação baseado em crítica}
\label{exemple-critiquing}

Com o intuito de não alterar o que já foi proposto no site da Liva, que faz o uso de uma ferramenta de busca baseada em preferência (Perfil de Busca) para recomendar propriedades, e ainda deixar o sistema de recomendação mais robusto seguindo uma abordagem, decidiu-se construir um sistema de recomendação baseado em crítica como descrito na seção \ref{Critiquing-based} seguindo a abordagem de crítica de exemplo.
	
Segundo Burke, Felfernig e Goker (\citeyear{Burke}), e como já dito anteriormente (seções \ref{Collaborativefiltering} e \ref{Contentbasedfiltering}), as abordagens tradicionais de recomendação, filtragem colaborativa e baseada em conteúdo, não são adequadas em situações de itens de alto valor, como propriedades ou veículos, que não são comprados com frequência. A abordagem baseada em conhecimento pode mitigar esse problema explorando requisitos explícitos do usuário, mas ela sofre de gargalos de aquisição de conhecimento associados aos esforços iniciais necessários para gerar o conhecimento do domínio.

Dessa forma, a abordagem baseada em crítica surgiu para ser uma eficiente tecnologia de recomendação baseada na preferência do usuário e evitar os problemas descritos anteriormente.

\subsubsection{Modelo da preferência}

Para o presente projeto decidiu-se por realizar a implementação da abordagem de crítica de exemplo, melhorando a ferramenta de busca baseada no ambiente virtual da Liva. A ferramenta de busca baseada em preferências é utilizada para obter as preferências iniciais dos usuários, enquanto a crítica de exemplo é uma abordagem que permite aos usuários aprimorar suas preferências para localizar o item ideal que atenda às suas necessidades.

O usuário segue um processo como demonstrado na Figura \ref{fig:exemple-critiquing}:

\begin{figure}[H]
    \centering
    \includegraphics[scale=0.5]{figuras/proposta/exemple-critiquing.png}
    \caption[Interação do usuário com a abordagem de crítica de exemplo]{Interação do usuário com a abordagem de crítica de exemplo. Fonte: \cite{Viappiani}. Traduzido.}
    \label{fig:exemple-critiquing}
\end{figure}

Após o usuário ter definido seu modelo de preferência inicial o sistema é capaz de apresentar exemplos que o usuário possa considerar. Esses exemplos são propriedades que são ótimas para a atual busca de preferência (Perfil de Busca) configurada. O usuário irá revisar seu modelo de preferência através de críticas sobre os exemplos apresentados. Quando o usuário identifica o item certo o processo termina.

O usuário irá apresentar suas preferências através do Perfil de Busca como na Figura \ref{fig:filtro_busca}. São comportados os seguintes atributos das propriedades: tipo de imóvel, faixa de preço, área, localização (Cidade e Bairro), quantidade de quartos, quantidade de banheiros e quantidade de vagas na garagem. Além disso, usuário seleciona a importância de cada atributo (peso de 1 a 5). O usuário irá criticar as propriedades de acordo com a Figura \ref{fig:prototipo_critico1}, em que ele pode postar críticas para a solução próxima, com as escolhas de manter ou melhorar algum atributo.

Como já demonstrado na seção \ref{Critiquing-based}, o modelo de preferência será baseado no MAUT (teoria da utilidade de atributos múltiplos), em que será computada a utilidade para cada propriedade e assim serem ordenadas e apresentadas.

Segundo Viappiani et al. (\citeyear{Viappiani}) na abordagem de crítica de exemplo, cada crítica pode ser considerada como uma restrição leve e o modelo de preferência pode ser desenvolvido simplesmente coletando críticas incrementalmente. Uma restrição leve é uma função de um atributo ou uma combinação de atributos para um número que indica o grau em que a restrição é violada.

Viappiani et al. (\citeyear{Viappiani}) dá o seguinte exemplo: um atributo que pode assumir os valores \textbf{a}, \textbf{b} e \textbf{c}, pode, para uma restrição leve, indicar preferência pelo valor b. Nesse caso haveria o mapeamento dos valores \textbf{a} e \textbf{c} para 1 e \textbf{b} para 0, indicando que somente \textbf{b} não quebrou a restrição. Uma preferência pela área de superfície de pelo menos 30 metros quadrados, onde uma pequena violação de até 5 metros quadrados pode ser aceitável, pode ser expressa por uma função linear por partes, como na Figura \ref{fig:funcao_linear}.

\begin{figure}[H]
    \centering
    \includegraphics[scale=0.8]{figuras/proposta/funcao_linear.png}
    \caption[Função linear por partes]{Função linear por partes. Fonte: \cite{Viappiani}.}
    \label{fig:funcao_linear}
\end{figure}

O autor considera que as restrições leves permitem que os usuários expressam preferências relativamente complexas de maneira intuitiva. Isso torna as restrições leves um forma útil para modelos de preferência de um \textit{example-critiquing} (crítica de exemplo).

\subsection{Banco de dados}

Como já descrito anteriormente serão utilizados dois tipos de banco de dados, um relacional com a tecnologia PostgreSQL e outro de valor-chave com Redis. Esses dois bancos são somente referente ao módulo de aprendizado de máquina, pois o módulo de crítica não necessita de armazenar dados.

Na base de dados relacional, serão armazenados dados para backup como o histórico de visualizações dos usuários (VIEW\_PROPERTY\_HISTORY - Histórico de propriedades visualizadas), histórico de buscas realizadas pelo usuário (SEARCH\_PROFILE\_HISTORY - Histórico de perfis de busca) e as propriedades (PROPERTIES). Além disso, são guardados dados de treino do modelo de aprendizado de máquina (TRAINING DATA).  Nas Figuras \ref{fig:der} e \ref{fig:diagrama_esquema} são apresentados os diagramas do banco de dados relacional.

\begin{figure}[H]
    \centering
    \includegraphics[scale=0.40]{figuras/proposta/der.png}
    \caption[Diagrama de Entidade-Relacionamento (DE-R)]{Diagrama de Entidade-Relacionamento (DE-R).}
    \label{fig:der}
\end{figure}

\begin{figure}[H]
    \centering
    \includegraphics[scale=0.40]{figuras/proposta/diagrama_esquema.png}
    \caption[Diagrama de Esquemas]{Diagrama de Esquemas.}
    \label{fig:diagrama_esquema}
\end{figure}

Inicialmente, a tabela SEARCH\_PROFILE\_HISTORY será populada com base nos perfis de buscas atuais dos usuários, adquiridos na base da dados da Liva. A tabela de propriedades em vermelho no diagrama da Figura \ref{fig:der} está destacada dessa forma porque ela não será armazenada no banco do sistema de recomendação mas, como já existente, na base da Liva. No banco de dados da Liva as propriedades são atualizadas diariamente para cada imobiliária existente. Como os dados vindos de cada imobiliária não seguem um padrão e muitas vezes há a existência de muitas características ausentes, foi decidido focar somente em características comuns e raramente faltantes como: tipo do imóvel, área total e útil, preço, localidade, quantidade de quartos, banheiros e suítes.

A tabela referente ao modelo de aprendizado de máquina será populada com dados que serão coletados do início do primeiro semestre de 2020 até a conclusão do sistema em junho, com o fim de se gerar um conjunto de dados (\textit{dataset}) de treino inicial.

Em relação ao banco de dados no Redis, como já apresentado na arquitetura anteriormente, existem quatro entidades como mostrado na Figura \ref{fig:redis}.

\begin{figure}[H]
    \centering
    \includegraphics[scale=0.4]{figuras/proposta/redis.png}
    \caption[Representação do banco de dados no Redis]{Representação do banco de dados no Redis.}
    \label{fig:redis}
\end{figure}

Os dados da entidade PROPRIEDADES serão atualizados constantemente pelo serviço descrito na arquitetura, \textit{Data Updater}. Ele irá atualizar os dados a partir da leitura da tabela PROPERTIES concentrada no banco de dados da Liva.

\section{Suporte Tecnológico}

\subsection{Python}

Python é uma linguagem de programação de alto nível com o paradigma OO (orientada a objeto), sendo ela uma linguagem interpretada e não compilada, como por exemplo, a linguagem C. Conhecida também por ter uma sintaxe muito clara e exigir poucas linhas de código comparadas a outras linguagens de programação a linguagem Python é muito utilizada em vários serviços de grandes empresas, sendo uma linguagem formada por uma grande comunidade \cite{Python:2019}.

A versão que será implementada no serviço “Sistema de Recomendação API” é a 3.7.3.

\subsection{Django}

O Django é um \textit{framework} sob a licença BSD, criado para a linguagem python que usa o padrão MVT (\textit{Model-View-Template}), na qual é uma adaptação do clássico padrão MVC (\textit{Model-View-Controller}). Essa tecnologia visa o desenvolvimento rápido, pois é baseada no princípio  DRY (\textit{Don’t Repeat Yourself}), em que a ideia principal é evitar ao máximo o retrabalho no desenvolvimento de software \cite{Django:2019}.

Ele também já cuida de várias rotinas comuns de aplicações \textit{Web}, como por exemplo: servidor, conexão com o banco de dados, organização de rotas (\textit{endpoints}), autenticação entre outras rotinas. Contém também a tecnologia ORM (\textit{Object-relational mapping}), que tem o intuito de mapear cada atributo e relacionamento das classes para o banco de dados, ou seja, não é necessário criar \textit{scripts} SQL (\textit{Structured Query Language}) na mão para tal. 

A versão que será usada no projeto para o serviço “Sistema de Recomendação API” é a 2.2.

\subsection{Django REST \textit{Framework}}

Para construção da API (Interface de Programação de Aplicações) com o paradigma arquitetural REST será utilizado a ferramenta denominada Django REST \textit{framework} que disponibiliza diversas ferramentas para construção da mesma, sendo que ela ainda possibilita algumas funcionalidades extras, como o \textit{serializer}, que permite a conversão de tipos de dado como JSON e XML para os tipos de dados nativos do python, e vice e versa \cite{DjangoRest:2019}.

A respeito dessa ferramenta, ela usa o padrão MV (\textit{Model-View}), pois por ser uma API, não há o \textit{Template} que teria o papel de um \textit{front-end}, como no Django. Essa tecnologia também possui uma grande comunidade, excelente documentação para consulta e é altamente utilizada por grandes empresas, como:  Mozilla , Red Hat , Heroku e Eventbrite \cite{DjangoRest:2019}.

A versão que será usada para implementação do serviço “Sistema de Recomendação API” será a 3.10.3.

\subsection{Scikit-Learn}
\label{scikit_learn}

O Scikit-Learn é um módulo do python que possui vários algoritmos de aprendizagem de máquina (por exemplo \textit{random forest e gradient boosting}) para problemas supervisionados e não supervisionados. Está biblioteca é construída em Numpy e Scipy e tem como objetivo trazer conhecimento de aprendizado de máquina para os usuários não especialistas usando uma linguagem de alto nível (Python). Um dos destaques dessa tecnologia está na facilidade de uso, no desempenho e na documentação \cite{PREDEGOSA:2011}.

A versão que será usada para implementação será a 0.21.3.

\subsection{Pandas}

Pandas é uma biblioteca do python de código aberto com a licença BSD, que preza em um bom desempenho e facilidade de uso. O propósito deste módulo é realizar análise e modelagem de dados (bem parecido com a linguagem R) usando o objeto conhecido como “\textit{DataFrame}” (Quadro de dados), podendo ler e escrever dados no formato CSV (valores separados por vírgula), texto, SQL (\textit{Structured Query Language} - Linguagem de consulta estruturada), HDF5 e Excel \cite{pandas:2019}.

A versão que será usada para implementação será a 0.3.1.

\subsection{Numpy}

Numpy é um pacote fundamental do python com a licença BSD (Berkeley \textit{Software Distribution}) quando se trata de Ciência de Dados que adiciona um poderoso objeto com matriz N-dimensional, podendo integrar com código C, C++ e Fortran, além de possuir recursos importantes de álgebra linear (transformação de Fourier, números aleatórios, por exemplo) \cite{numpy:2019}.

A versão que será usada para implementação será a 1.17.3.

\subsection{Pickle}

O Pickle é um módulo do python que implementa protocolos binários com a finalidade de serializar e desserializar uma estrutura de objetos dessa mesma linguagem. O “\textit{Pickling}” significa transformar um objeto da hierarquia do python para fluxo em bytes, já o desserializar é o fluxo contrário a isso \cite{pickle:2019}.

A versão que será usada para implementação será a 5.

\subsection{Xgboost}

Xgboost é uma biblioteca de licença Apache que proporciona o \textit{gradient boosting} com suporte para C++, Python, R e outras linguagens de programação. Criada para ser altamente eficiente, flexível e portátil, resolvendo vários problemas de Ciência de Dados utilizando algoritmos de aprendizagem de máquina sob a estrutura \textit{gradient boosting} \cite{XGBOOST:2019}.

A versão que será utilizada no serviço “Sistema de Recomendação API” será a 0.90.

\subsection{JavaScript}

JavaScript é uma linguagem de programação interpretada (assim como o python) muito utilizada no lado do cliente, mas também podendo ser usado do lado do servidor para tratamento de dados. Sendo a linguagem mais usada atualmente no lado do cliente, ou seja, no navegador do usuário, juntamente com HTML5 e CSS3, deixando as páginas Web estáticas, agora interativas. Com o avanço da linguagem e a criação do \textit{framework} NodeJS, atualmente JavaScript é muito usado no \textit{back-end} dos serviços \cite{JavaScript:2019}.

A versão que será utilizada para a implementação dos serviços “Data Updater” será a EcmaScript 6.

\subsection{NodeJS}

O NodeJS é definido como um “ambiente de execução Javascript do lado do servidor”, ou seja, é um \textit{framework} orientado a eventos baseado na linguagem Javascript que possibilita a criação de aplicações não dependentes de um navegador. Um ponto positivo que levou a sua alta adoção pelos programadores é a sua alta escalabilidade, além de sua arquitetura flexível e sendo um ótimo aliado a arquitetura de microsserviços \cite{nodejs:2019}.

Um dos diferenciais do NodeJS é o gerenciador de pacotes que ele tem, o NPM, que contém o maior repositório de software do mundo, sendo que um dos pacotes é o Express, na qual é um \textit{framework} para desenvolvimento \textit{Web} \cite{lenon:2018}.

A versão que será utilizada para a implementação do serviço “Data Updater” será a 13.1.0.

\subsection{ReactJS}

ReactJS é um biblioteca Javascript orientada a componentes que tem como objetivo criar interfaces interativas de usuário sem muita complexidade. Cada componente tem seu próprio estado chamado de “\textit{state}” e a cada mudança desse estado, o React atualiza e renderiza quase instantaneamente aquele componente. Essa tecnologia é compatível com a sintaxe JSX, mas sendo opcional e não obrigatório seu uso pelo programador \cite{reactjs:2019}.

A versão já utilizada no serviço de \textit{front-end} da Liva e que também será utilizada para implementação dos novos componentes do sistema de recomendação será a 16.11.

\subsection{Bootstrap}

Bootstrap é uma ferramenta de código livre para desenvolvimento de sites e aplicativos web que tem HTML, CSS e JavaScript. Sendo um dos \textit{frameworks} mais famosos do mundo usado para criar ambientes virtuais responsivos e móveis (\textit{mobile}), disponibilizando componentes para web (Internet) com estilos prontos e \textit{templates} de \textit{sites} \cite{bootstrap:2019}.

A versão já usada no serviço de \textit{front-end} da Liva e que também será utilizada para implementação dos novos componentes do sistema de recomendação será a 4.3.

\subsection{Ruby}

Ruby é uma linguagem de programação de código totalmente livre orientada a objeto, tal qual que tudo em ruby é objeto, pois cada parte do código tem suas propriedades e ações relacionadas a aquele tipo de objeto. O Ruby também é flexível, ou seja, os utilizadores podem remover e redefinir partes da linguagem a vontade. Além disso, o Ruby usa o padrão “\textit{sugar syntax}” (açúcar sintático) que facilita lembrar de como escrever o código \cite{ruby:2019}.

A versão já usada no serviço \textit{back-end} “Liva API” é a 2.6.5.

\subsection{Rails}

Rails é um meta-framework (um framework compostos de outros frameworks) feito com a linguagem Ruby e também é de código aberto sob a licença MIT, assim como a própria linguagem. O padrão arquitetural seguido por essa tecnologia é o clássico MVC (Model-View-Controller), e também os padrões Web, como JSON e XML. A ideia principal do Rails é aumentar a velocidade de desenvolver uma aplicação Web \cite{portalgsti:2019}.

A versão já usada no serviço \textit{back-end} “Liva API” é a 6.0.1.

\subsection{PostgreSQL}

PostgreSQL é uma famoso banco relacional de código aberto que usa a linguagem de \textit{script} SQL. Ele também contém vários recursos para a escala e a segurança das cargas de trabalho de dados. Esse banco ganhou sua reputação através da sua arquitetura robusta, confiabilidade, integridade, um conjunto de recursos e a dedicação de sua comunidade. Atualmente, ele é o banco relacional de código aberto preferido pelos desenvolvedores \cite{postgres:2019}.
Fora essas características, o PostgreSQL é altamente extensível, no sentindo do desenvolvedor poder criar seus próprios tipos de dados e funções personalizadas \cite{postgres:2019}.

A versão que já está sendo usada como banco de dados para o serviço “Liva API” e que será usado nos outros banco de dados será á 12.0.

\subsection{Redis}

O Redis é uma base de dados em memória com licença BSD, ou seja, tem código aberto. Muito utilizado como cache e banco de dados em serviços, suportando vários tipos de dados, como: \textit{strings, hashes, lists, sets, sorted sets} dentre outros. Ele permite realizar algumas operações atômicas com os dados, por exemplo, inserir em uma \textit{string}, incrementar algum valor numa \textit{hash}, inserir elementos numa \textit{list} e outros operações \cite{redis:2019}.

O responsável pelo seu excelente desempenho é o conjunto de dados na memória, mas caso seja necessário é capaz de gravar em disco ou anexar informações em \textit{log} \cite{redis:2019}.

A versão que será utilizada para se conectar com o serviço “\textit{Data Updater}” e o “Sistema de Recomendação API” é a 5.0.

\subsection{Docker}

Docker é uma ferramenta de virtualização muito usada em vários serviços, na qual foi projetada para facilitar a criação e a implantação usando os chamados “contêineres” de modo que cada \textit{container} pode ser feito um empacotamento de algum determinado serviço com todos os pacotes e configurações necessárias para a execução dele \cite{OPENSOURCE.COM:2019}.

O Docker disponibiliza uma grande quantidade de imagens no repositório chamado DockerHub, sejam elas oficiais ou feitas por desenvolvedores ao redor do mundo. Essas imagens contém Linux, possibilitando a base para criação da virtualização da aplicação \cite{DOCKERHUB:2019}.

A versão que será usada para virtualização dos serviços “\textit{Data Updater}” e “Sistema de Recomendação API” é a 19.03.

\subsection{Amazon \textit{Elastic Container Service}}

O Amazon ECS como é conhecido, é uma plataforma de gerenciamento de contêineres rápida e dimensível que possibilita o gerenciamento de contêineres dentro de um \textit{cluster}. O usuário é capaz de iniciar ou parar aplicações que rodam nesses contêineres com um simples comando, inclusive disponibiliza o estado de cada serviço rodando dentro do \textit{cluster} \cite{amazonECS:2019}.

Ele dá a possibilidade do usuário escolher os recursos que deseja para rodar determinado serviço e sem se preocupar com as configurações de cada recurso ou com a escalabilidade da infraestrutura de gerenciamento \cite{amazonECS:2019}.

\subsection{Amazon Kinesis}

O Amazon Kinesis é um serviço que serve para coleta, análise e transmissão (\textit{streaming}) de grandes quantidades de dados em tempo real. O Kinesis pode ser executado em instâncias do Amazon EC2 com a biblioteca Kinesis Client. Essa tecnologia possibilita enviar os dados processados para vários outros serviços AWS \cite{KINESIS:2019}.

\subsection{Amazon S3}

O  Amazon Simple Storage Service (Amazon S3) é um serviço de armazenamento de dados junto com escalabilidade, durabilidade, segurança e performance, ou seja, é possível armazenar qualquer volume de dados em uma grande escala de variabilidade. O gerenciamento e a organização dos dados é feito de maneira fácil \cite{S3:2019}.

\subsection{Git}

Git é uma ferramenta de código aberto de controle de versão planejado para pequenos até grandes projetos com rapidez. O seu diferencial comparado a outras ferramentas de versionamento é a sua estrutura de \textit{branches} (ramificações) e \textit{mergings} (fundição), na qual é possível criar várias \textit{branches} locais totalmente independentes uma da outra, além da criação e da exclusão ser fácil e rápida \cite{git:2019}. O repositório de código escolhido será o GitHub.

A versão que será usada nos serviços “\textit{Data Updater}” e “Sistema de Recomendação API” será a 2.24.

\subsection{ZenHub}
\label{section_zenhub}

ZenHub é uma ferramenta de gerenciamento de projetos ágeis para desenvolvedores de software integrado a plataforma do GitHub. Usando as informações do projeto trazidas do próprio GitHub. Essa ferramenta é capaz de gerar gráficos, relatórios, fluxo de trabalho dos desenvolvedores e andamento do projeto \cite{zenhub:2019}.
