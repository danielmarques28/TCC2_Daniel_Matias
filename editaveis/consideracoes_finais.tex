\chapter{Considerações finais}

Esse último capítulo apresenta os resultados atingidos da solução depois do fluxo de desenvolvimento, além da avaliação do mesmo no qual foi analisado se o objetivo proposto foi alcançado. Além disso, será sugerido possíveis trabalhos futuros que podem ser desenvolvidos a partir dessa solução.

\section{Conclusão}

Para alcançar os objetivos descritos na seção \ref{section:objetivos}, que referem-se ao desenvolvimento de um sistema de recomendação, foi realizada inicialmente uma pesquisa bibliográfica necessária, além da realização da documentação referente a arquitetura do projeto. Dessa forma, a descrição de como seria todo funcionamento do sistema proposto e seu processo de desenvolvimento a partir da análise e planejamento, projeto e desenvolvimento envolvendo a metodologia adequada aos objetivos almejados.

Algumas mudanças na fase de desenvolvimento foram necessárias como foi descrito no capitulo anterior (\ref{desenvolvimento}), para assim, poder atingir os objetivos e entregar uma melhor qualidade no final do processo.

O objetivo principal desse trabalho foi a construção de um sistema de recomendação que funcionaria como um serviço mais interessante para o ambiente virtual da Liva.vc, podendo ele proporcionar ao usuário, possível cliente, uma melhor experiência de uso com recomendações mais coerentes com as demandas e preferências (re)conhecidas.

Após a fase de desenvolvimento (capítulo \ref{desenvolvimento}), foi observado que o sistema de recomendação elaborado conseguiu contribuir para agregar um maior valor aos clientes do sistema da Liva a partir de análises de métricas, na qual é detalhado no tópico de analise (\ref{analise_gerais}).

Foi constatado que os objetivos indicados na seção \ref{section:objetivos} foram atingidos no final de todo o processo, como a construção de todo o sistema de recomendação, a base de dados robusta para o modelo de aprendizado de máquina e a análise final em cima das métricas.

Para cada usuário, o sistema de recomendação baseado em aprendizado de máquina ira sugerir imóveis semelhantes com o seu perfil, utilizando-se de características de imóveis em que o usuário estava navegando e em seu filtro de busca. O outro módulo do sistema de recomendação implementado é baseado na abordagem de crítica, na qual o usuário é capaz de "criticar" o imóvel sendo visualizado e utilizar também uma ferramenta de busca de imóveis, em que ele pode colocar o nível de importância em diferentes atributos de propriedade.

Portanto, agora fica claro o beneficio gerado pelo sistema de recomendação implantado, em que ao usuário entrar no sistema da Liva e navegar, poderá ver recomendações de propriedades, facilitando mais sua pesquisa por um imóvel ideal, diminuindo o tempo para encontra-lo. A cada imóvel que o usuário navega, melhorará a precisão das sugestões do sistema de recomendação para aquele determinado usuário, e quanto mais usuários utilizarem a ferramenta, melhores serão as recomendações geradas.

\section{Trabalhos futuros}

Esse trabalho tinha como escopo desenvolver um sistema de recomendação híbrido para uma plataforma \textit{e-commerce} da empresa Liva no ramo imobiliário, a fim de melhorar o serviço já utilizado por seus clientes, recomendando imóveis parecidos com o perfil de cada usuário a partir de dois módulos de recomendação desenvolvidos (baseado em crítica e baseado em aprendizado de máquina).

No capítulo antecedente (\ref{desenvolvimento}), foi detalhado que era necessário realizar alguns modificações na proposta inicial do trabalho para melhor alcançar os objetivos propostos. Ainda assim, é necessário continuar manutenindo a aplicação, sendo essa manutenção feita através de possíveis evoluções (trabalhos futuros) e possíveis correções que será necessário realizar por algum motivo, seja de mudança de regra de negócio ou mudanças nas tecnologias utilizadas no desenvolvimento.

Um dos fundamentos ensinado no curso de Engenharia de Software é pensar na manutenção e evolução da aplicação, em outras palavras, corrigir futuros problemas e desenvolver novas funcionalidades. Pensando nisso, foram levantadas algumas \textit{features} em que aumentariam o valor de negócio do sistema de recomendação proposto, e estas funcionalidades seriam:

\begin{itemize}
  \item O componente de crítica (figura \ref{fig:componente_critica2}) apresentará dicas de valores para cada descrição do imóvel no componente. Isso seria, por exemplo, apresentar valores de preços alternativos à aquele imóvel sendo visualizado, com base em outros imóveis parecidos naquela mesma localização. Outro exemplo, seria exibir um valor de quarto compatível com aquele preço selecionado.

  \item Detectar e tratar de forma automatizada os casos de usuários com um número muito alto de cliques no modelo, com o propósito de evitar \textit{overfitting} (sobreajuste) no modelo treinado que ocasiona em má predição de novos resultados.
  
  \item Foi possível observar uma tendência aos dados de predição com CV = 1 aumentarem mais de acordo com o tempo, isto significa que pode haver um desequilíbrio entre a quantidade de dados com CV = 1 e CV = 0 no futuro. Caso realmente haja, seria possível equilibrar esses dados da mesma forma que foi dito anteriormente no tópico \ref{design_modelo}, na qual para cada uma linha com CV = 1 será gerado 10 linhas com CV = 0.
\end{itemize}

Além dessas \textit{features} descritas, outras tipos de funcionalidades que também agregariam valor ao sistema são bem-vindas, seja melhorando a usabilidade e experiência do usuário, ou seja ampliando a probabilidade de predição em diferentes tipos de perfis de usuário e suas localidades.

%1 - Algoritmo funcionar para equilibrar os dados de predição de \textit{dislikes} e ficarem 1-10 como feito no \textit{dataset} de \textit{lead}

%2 - Melhor controle nos cliques dos usuários no modelo para não dar \textit{overfitting}

%3 - Apresentação de dicas nos campos de criticas
