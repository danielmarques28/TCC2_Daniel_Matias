\chapter{Considerações finais}

Esse último capítulo apresenta os resultados atingidos da solução depois do fluxo de desenvolvimento, além da avaliação do mesmo no qual foi analisado se o objetivo proposto foi alcançado. Além disso, será sugerido possíveis trabalhos futuros que podem ser desenvolvidos a partir dessa solução.

\section{Conclusão}

Para alcançar os objetivos descritos na seção \ref{section:objetivos}, que referem-se ao desenvolvimento de um sistema de recomendação, foi realizado inicialmente um levantamento bibliográfica acerca do tema, além da realização da documentação referente a arquitetura do projeto. Dessa forma, a descrição de como seria todo funcionamento do sistema proposto e seu processo de desenvolvimento a partir da análise e planejamento, projeto e desenvolvimento envolvendo a metodologia adequada aos objetivos almejados.

Algumas mudanças na fase de desenvolvimento foram necessárias como foi descrito no capitulo \ref{desenvolvimento}, para assim poder atingir os objetivos com melhor qualidade no processo.

O objetivo principal desse trabalho foi a construção de um sistema de recomendação que funcionaria como um serviço mais interessante para o ambiente virtual da Liva.vc, podendo ele proporcionar ao usuário, possível cliente, uma melhor experiência de uso com recomendações mais coerentes com as demandas e preferências (re)conhecidas.

Após a fase de desenvolvimento (capítulo \ref{desenvolvimento}) foi observado que o sistema de recomendação elaborado conseguiu contribuir para agregar um maior valor aos clientes do sistema da Liva. Essa observação foi possível através de cálculos e analises de métricas como a CVR (Conversion Rate).

Ao coletar dados do sistema de recomendação elaborado em ambiente de produção por aproximadamente um mês juntamente com o período anterior de mesma duração, foi possível fazer o cálculo da métrica CVR, que se refere a taxa de conversões de imóveis dentre as recomendações geradas para os usuários. Dessa forma, foi possível comparar o CVR para os dois períodos que demonstrou uma melhora de 11\% para o período do novo sistema de recomendação elaborado. Além disso, foi comparado também a taxa de usuários ativos para os diferentes períodos, na qual demonstrou uma proporção parecida, conferindo maior credibilidade para a métrica CVR.

Foi constatado que os objetivos indicados na seção \ref{section:objetivos} foram atingidos no final de todo o processo, como a construção de todo o sistema de recomendação, a base de dados robusta para o modelo de aprendizado de máquina e a análise final em cima das métricas.

Para cada usuário, o sistema de recomendação baseado em aprendizado de máquina irá sugerir imóveis semelhantes com o seu perfil, utilizando-se de características de imóveis em que o usuário estava navegando e em seu filtro de busca. O outro módulo do sistema de recomendação implementado é baseado na abordagem de crítica, na qual o usuário é capaz de "criticar" o imóvel sendo visualizado e utilizar também uma ferramenta de busca de imóveis, em que ele pode colocar o nível de importância em diferentes atributos de propriedade.

Dessa forma, é possível observa o benefício gerado pelo novo  sistema de recomendação implantado, em que o usuário ao entrar e navegar no sistema da Liva poderá ver recomendações de propriedades, facilitando mais sua pesquisa por um imóvel ideal, aumentando as chances de encontrá- lo e diminuindo o tempo de busca. A cada imóvel que o usuário navega melhorará a precisão das sugestões do sistema de recomendação para aquele determinado usuário, e quanto mais usuários utilizarem a ferramenta, melhores serão as recomendações geradas.

\section{Trabalhos futuros}

Esse trabalho tinha como escopo desenvolver um sistema de recomendação híbrido para uma plataforma de \textit{e-commerce} para a empresa Liva no ramo imobiliário, a fim de melhorar o serviço já utilizado por seus clientes, recomendando imóveis parecidos com o perfil de cada usuário a partir de dois módulos de recomendação desenvolvidos (baseado em crítica e baseado em aprendizado de máquina). No capítulo \ref{desenvolvimento} foram detalhadas as mudanças necessárias a partir das modificações na proposta inicial do trabalho para melhor alcançar os objetivos propostos.

No sentido de sempre continuar melhorando o software é necessário prosseguir com novas evoluções com o intuito de trazer melhorias para o usuário final. Com isso, foram levantadas algumas \textit{features} em que aumentariam o valor de negócio do sistema de recomendação proposto. Estas novas \textit{features} seriam:

\begin{itemize}
  \item Na parte de alterar os valores das características do imóvel no componente de crítica (Figura \ref{fig:componente_critica2}), apresentaria dicas de possíveis valores para cada atributo do imóvel, com base em outros imóveis parecidos com aquele que está sendo visualizado, ou seja, ao usuário criticar o preço de alguma propriedade, o sistema irá comparar esse imóvel com outros imóveis da mesma localidade e com valores dos atributos parecidos. Assim poderá apresentar para o usuário possíveis valores de preço para aquele domínio, sem ele precisar saber qual seria a faixa de preço para aquele tipo de imóvel que está sendo procurado. Outro exemplo, seria exibir outros possíveis valores de quarto, banheiro, suíte ou vagas de garagem compatível.

%   \item Detectar e tratar de forma automatizada os casos de usuários com um número muito alto de cliques no modelo, com o propósito de evitar \textit{overfitting} (sobreajuste) no modelo treinado que ocasiona em má predição de novos resultados.
 
 ⦁	Um módulo de qualificação do cliente poderia ser agregado no modelo de aprendizado de máquina gerando novas \textit{features}, em que se poderia obter, por meio de formulários ao cliente, como por exemplo com perguntas do tipo: "Utilizará FGTS na compra do imóvel?". Dessa forma, será possível determinar o momento de compra do cliente gerando novas \textit{features} para o modelo e assim melhorando sua eficácia. Outra possibilidade de novas \textit{features} seria relacionado as ações dos corretores, como por exemplo possíveis imóveis recomendados diretamente do corretor, ou um \textit{feedback} do corretor em relação ao cliente.
  
%   \item Foi possível observar uma tendência aos dados de predição com CV = 1 aumentarem mais de acordo com o tempo, isto significa que pode haver um desequilíbrio entre a quantidade de dados com CV = 1 e CV = 0 no futuro. Caso realmente haja, seria possível equilibrar esses dados da mesma forma que foi dito anteriormente no tópico \ref{design_modelo}, na qual para cada uma linha com CV = 1 será gerado 10 linhas com CV = 0.
% \end{itemize}

\item Mais qualificações em cima das propriedades, criando algoritmos para gerar \textit{features} com \textit{tags} de localidade, como por exemplo se há hospitais, mercados e parques próximos ao imóvel ou características do bairro, se é mais seguro, ou ainda ter características como personalidades de pessoas que moram próximas. Todas essas características agregariam ainda mais no perfil do cliente e ajudaria a encontrar melhores recomendações.
\end{itemize}

% Além dessas \textit{features} descritas, outras tipos de funcionalidades que também agregariam valor ao sistema são bem-vindas, seja melhorando a usabilidade e experiência do usuário, ou seja ampliando a probabilidade de predição em diferentes tipos de perfis de usuário e suas localidades.

%1 - Algoritmo funcionar para equilibrar os dados de predição de \textit{dislikes} e ficarem 1-10 como feito no \textit{dataset} de \textit{lead}

%2 - Melhor controle nos cliques dos usuários no modelo para não dar \textit{overfitting}

%3 - Apresentação de dicas nos campos de criticas
